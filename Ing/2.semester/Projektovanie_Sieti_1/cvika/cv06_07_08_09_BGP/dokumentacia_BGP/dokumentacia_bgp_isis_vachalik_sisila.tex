\documentclass[12pt,twoside,a4paper]{report}
\usepackage[a4paper,width=150mm,top=25mm,bottom=25mm,bindingoffset=6mm]{geometry}

\usepackage[utf8x]{inputenc}
\usepackage[slovak]{babel}
\usepackage{palatino,verbatim}

% Balicek pre priamu rec - \say
\usepackage{dirtytalk}

% Balicek "alltt" je to iste ako "verbatim" mod, ale navyse podporuje aj formatovacie znacky textu
\usepackage{alltt}

% Obrazky
\usepackage{graphicx}
\graphicspath{ {obr/} }

% Cislovanie obrazkov a tabuliek
\usepackage{chngcntr}
%Cisluj obrazky nezavisle od cisla kapitol/podkapitol
\counterwithout{figure}{subsection}
\counterwithout{table}{subsection}

% Referencovanie kapitol/sekcii/... podľa ich nadpisu
\usepackage{nameref}

% Tabulky s viacriadkovymi bunkami a zlucenymi bunkami
% Tabulky generujem naastrojom "http://www.tablesgenerator.com/"
\usepackage{booktabs}
\usepackage{multirow}
% LaTeX ma problemy s prikazmi cline a cmidrule, ked je babel nastaveny na slovencinu/cestinu, kvoli definicii pomlcky
% NAMIESTO POMLCKY POUZI ZNAK ZNAMIENKA MINUS "−" (plati hlavne v nazvoch nadpisov a labelov)
\usepackage{etoolbox}
\preto\tabular{\shorthandoff{-}}

%Uloz obrazok tam, kde je deklarovany
%\usepackage[subsection]{placeins}

\newcommand\sktxt[1]{\foreignlanguage{slovak}{#1}}

\begin{document}
\pagenumbering{arabic}

\setcounter{chapter}{1}
\chapter*{Internet Peering}
\paragraph{}
Andrej Šišila, Marián Vachalík

\tableofcontents

\newpage
\section{Topológia}
\paragraph{}
Budeme konfigurovať smerovacie protokoly BGP a IS-IS na topológií, ktorá je znázornená na obrázku \ref{fig:bgp_isis_topo}. Vnútri autonómnych systémov sme konfigurovali smerovacie protokoly IS-IS a BGP (iBGP). Medzi autonómnymi systémami sme konfigurovali len BGP (eBGP). IP adresácia je uvedená v tabuľke \ref{tab:ip_adresacia} a dopĺňa grafické znázornenie topológie na obrázku \ref{fig:bgp_isis_topo}. Sieť medzi smerovačmi R1 a R5 nemá mať masku \say{/48} ale \say{/30}. Subrozhranie “.13” (VLAN 13) sme premenovali na “.23” (VLAN 23), lebo sieť je medzi smerovačmi R2 a R3 (23), a nie medzi R1 a R3 (13).

\begin{figure}[!htbp]
\centering
\includegraphics[width=14cm,keepaspectratio]{bgp_isis_topo}
\caption{Topológia BGP}
\label{fig:bgp_isis_topo}
\end{figure}



\begin{table}[!htbp]
\centering
\caption{IP adresácia}
\label{tab:ip_adresacia}
\begin{tabular}{|c|l|l|l|}
\hline
\textbf{Smerovač}    & \multicolumn{1}{c|}{\textbf{Rozhranie}} & \multicolumn{1}{c|}{\textbf{IP adresa}} & \multicolumn{1}{c|}{\textbf{Maska}} \\ \hline
\multirow{5}{*}{R1}  & Fa0/0                                   & 200.110.255.249                         & 255.255.255.252                     \\ \cline{2-4} 
                     & Fa0/1                                   & 64.34.255.253                           & 255.255.255.252                     \\ \cline{2-4} 
                     & S1/0                                    & 200.33.255.253                          & 255.255.255.252                     \\ \cline{2-4} 
                     & Lo0                                     & 10.255.255.1                            & 255.255.255.0                       \\ \cline{2-4} 
                     & Lo100                                   & 64.34.1.1                               & 255.255.255.128                     \\ \hline
\multirow{6}{*}{R2}  & Fa0/0                                   & 200.110.255.250                         & 255.255.255.252                     \\ \cline{2-4} 
                     & Fa0/1.23                                & 10.110.23.2                             & 255.255.255.0                       \\ \cline{2-4} 
                     & Fa0/1.24                                & 10.110.24.2                             & 255.255.255.0                       \\ \cline{2-4} 
                     & S1/0                                    & 200.110.255.253                         & 255.255.255.252                     \\ \cline{2-4} 
                     & Lo0                                     & 10.255.255.2                            & 255.255.255.0                       \\ \cline{2-4} 
                     & Lo100                                   & 200.110.0.2                             & 255.255.255.128                     \\ \hline
\multirow{5}{*}{R3}  & Fa0/0                                   & 200.110.255.241                         & 255.255.255.252                     \\ \cline{2-4} 
                     & Fa0/1.23                                & 10.110.23.3                             & 255.255.255.0                       \\ \cline{2-4} 
                     & Fa0/1.34                                & 10.110.34.3                             & 255.255.255.0                       \\ \cline{2-4} 
                     & Lo0                                     & 10.255.255.3                            & 255.255.255.0                       \\ \cline{2-4} 
                     & Lo100                                   & 200.110.0.133                           & 255.255.255.128                     \\ \hline
\multirow{6}{*}{R4}  & Fa0/0                                   & 200.110.255.237                         & 255.255.255.252                     \\ \cline{2-4} 
                     & Fa0/1.24                                & 10.110.24.4                             & 255.255.255.0                       \\ \cline{2-4} 
                     & Fa0/1.34                                & 10.110.34.4                             & 255.255.255.0                       \\ \cline{2-4} 
                     & S1/0                                    & 200.110.255.245                         & 255.255.255.252                     \\ \cline{2-4} 
                     & Lo0                                     & 10.255.255.4                            & 255.255.255.0                       \\ \cline{2-4} 
                     & Lo100                                   & 200.110.1.4                             & 255.255.255.128                     \\ \hline
\multirow{5}{*}{R5}  & Fa0/0                                   & 200.33.255.249                          & 255.255.255.252                     \\ \cline{2-4} 
                     & Fa0/1                                   & 10.100.15.2                             & 255.255.255.252                     \\ \cline{2-4} 
                     & S1/0                                    & 200.110.255.254                         & 255.255.255.252                     \\ \cline{2-4} 
                     & Lo0                                     & 10.255.255.5                            & 255.255.255.0                       \\ \cline{2-4} 
                     & Lo100                                   & 128.45.5.5                              & 255.255.255.128                     \\ \hline
\multirow{5}{*}{R6}  & Fa0/0                                   & 200.33.255.250                          & 255.255.255.252                     \\ \cline{2-4} 
                     & Fa0/1                                   & 10.110.67.6                             & 255.255.255.0                       \\ \cline{2-4} 
                     & S1/0                                    & 200.33.255.254                          & 255.255.255.252                     \\ \cline{2-4} 
                     & Lo0                                     & 10.255.255.6                            & 255.255.255.0                       \\ \cline{2-4} 
                     & Lo100                                   & 200.33.6.6                              & 255.255.255.128                     \\ \hline
\multirow{4}{*}{R7}  & Fa0/1                                   & 10.110.67.7                             & 255.255.255.0                       \\ \cline{2-4} 
                     & S1/1                                    & 200.33.255.245                          & 255.255.255.252                     \\ \cline{2-4} 
                     & Lo0                                     & 10.255.255.7                            & 255.255.255.0                       \\ \cline{2-4} 
                     & Lo100                                   & 200.33.7.7                              & 255.255.255.128                     \\ \hline
\multirow{4}{*}{R8}  & Fa0/0                                   & 200.110.255.242                         & 255.255.255.252                     \\ \cline{2-4} 
                     & Fa0/1                                   & 10.110.89.8                             & 255.255.255.0                       \\ \cline{2-4} 
                     & Lo0                                     & 10.255.255.8                            & 255.255.255.0                       \\ \cline{2-4} 
                     & Lo100                                   & 200.110.12.8                            & 255.255.255.128                     \\ \hline
\multirow{4}{*}{R9}  & Fa0/0                                   & 200.110.255.238                         & 255.255.255.252                     \\ \cline{2-4} 
                     & Fa0/1                                   & 10.110.89.9                             & 255.255.255.0                       \\ \cline{2-4} 
                     & Lo0                                     & 10.255.255.9                            & 255.255.255.0                       \\ \cline{2-4} 
                     & Lo100                                   & 200.110.13.9                            & 255.255.255.128                     \\ \hline
\multirow{4}{*}{R10} & S1/0                                    & 200.110.255.246                         & 255.255.255.252                     \\ \cline{2-4} 
                     & S1/1                                    & 200.33.255.246                          & 255.255.255.252                     \\ \cline{2-4} 
                     & Lo0                                     & 10.255.255.10                           & 255.255.255.0                       \\ \cline{2-4} 
                     & Lo100                                   & 223.255.255.10                          & 255.255.255.128                     \\ \hline
\end{tabular}
\end{table}


% Novu kapitolu davam na novu stranu, lebo bez toho mi zobrazuje tabulku v dalsej kapitole, kde ale tabulka nepatri.
\newpage

\section{Úlohy}
\subsection{Použiť IGP IS−IS (L2 only) single area dizajn, priame p2p prepojenia}
\subsubsection{Popis}
\paragraph{}
ISP1, ISP2 a Zákazník 1 používajú vnútorný smerovací protokol IS-IS.


\subsubsection{Konfigurácia}
\paragraph{}
Konfigurovali sme smerovače R2, R3, R4, R6, R7, R8 a R9. Nižšie uvádzame príklad konfigurácie pre R2.

\noindent
{\fontfamily{qcr}\selectfont
\begin{small}
\begin{alltt}
!R2
ena
conf t
hostname R2
no ip domain-lookup
username admin privil 15 secret admin
line con 0
  login local
  logging syn
  exec-time 120
line vty 0 15
  privilege level 15
  no login
int f0/0
  ip addr 200.110.255.250 255.255.255.252
  no shut
int f0/1
  no ip add
  isis network point-to-point
  no sh
int f0/1.23
  encap dot1q 23
  ip addr 10.110.23.2 255.255.255.0
  ip router isis
int f0/1.24
  encap dot1q 24
  ip addr 10.110.24.2 255.255.255.0
  ip router isis
int s1/0
  ip addr 200.110.255.253 255.255.255.252
  no shut
int lo0
  ip addr 10.255.255.2 255.255.255.255
  ip router isis
  no shut
int lo100
  ip addr 200.110.0.2 255.255.255.128
  ip router isis
  no shut
router isis
  net 49.0001.0102.5525.5002.00
  passive-interface lo0
  passive-interface lo100
  redistribute static
  is-type level-2
  metric-style wide
  exit
\end{alltt}
\end{small}
}

\paragraph{}
Najprv sme linky medzi autonómnymi systémami šírili cez IS-IS. Neskôr sa toto ukázalo ako nevhodné riešenie, pretože \say{flappovacie} linky u zákazníkov môžu spôsobiť nestabilitu siete. Preto boli rozhrania medzi AS odstranené z IS-IS príkazmi uvedenými nižšie.

\noindent
{\fontfamily{qcr}\selectfont
\begin{small}
\begin{alltt}
int <nazov_interfaceu>
  no ip router isis
router isis
  no passive-interface <nazov_interfaceu>
  no redistribute-connected
\end{alltt}
\end{small}
}


\subsubsection{Overenie}
\paragraph{}
Konfiguráciu IS-IS sme neoverovali. Kontrolu odstránenia liniek medzi autonómnymi systémami z IS-IS sme robili v kapitole \ref{sumarizacia} \nameref{sumarizacia}.






\subsection{Zabezpečiť plnú konektivitu prostredníctvom iBGP alebo eBGP protokolov pre zákaznícké a internetové smerovacie záznamy}
\subsubsection{Popis}
\paragraph{}
V rámci BGP sme každému smerovaču nakonfigurovali siete, s ktorými susedí resp. na ktoré je priamo pripojený príkazom \say{neighbor}. Podľa toho, či sa susediaca sieť nachádza v AS s rovnakým číslom ako je ASN daného smerovača, použije sa iBGP, inak sa použije eBGP. 

\paragraph{}
Na smerovačoch sme vytvorili dve virtuálne rozhrania: Loopback0 a Loopback100. Loopback0 mal IP adresu v tvare \say{10.255.255.X} s maskou \say{/32}, kde X je číslo smerovača. Všetky rozhrania Loopback100 majú masku \say{/25}. Router-ID sme nastavili na IP adresu rozhrania Loopback0. V rámci BGP sme ho nastavovali príkazom \say{bgp router-id 10.255.255.X}. Pokiaľ sa Router ID v BGP nenastaví hneď na začiatku, jeho zmena spôsobí rozpad BGP spojenia, ktoré sa po chvíli (rádovo v desiatkach sekúnd) obnoví. Loopback100 mal IP adresu z verejného rozsahu príslušnej autonómnej oblasti s maskou \say{/25}.

\paragraph{}
Siete obidvoch Loopback rozhraní sme ohlasovali príkazom \say{network} v rámci BGP. Potom sme pre ne použili príkaz \say{update-source}, ktorý slúži na prepísanie zdrojovej IP adresy na Loopback0. Keďže sme Loopback0 ohlásili príkazom \say{network}, susediace smerovače mu budú môcť odpovedať. Zdrojová adresa sa potom použije na otvorenie BGP spojenia medzi susednými smerovačmi. Loopback0 používame aj kvôli tomu, že bude vždy zapnutý, takže BGP spojenie bude stále aktívne.

\paragraph{}
Pokiaľ boli v AS viac ako dva smerovače, museli sme pridať príkaz \say{next-hop-self}, aby sme zaistili konektivitu medzi AS. Príkaz \say{next-hop-self} sa používa, keď sa BGP smerovač v jednom AS dozvie o ceste z iného AS cez eBGP a dá o tejto ceste vedieť zvyšným iBGP smerovačom v rámci svojho AS. Aby mohol iBGP smerovač získať konektivitu do tejto siete, použijeme na eBGP príkaz \say{next-hop-self}, ktorý namiesto toho, aby preposielal prefix s \say{next-hop} adresou zo siete medzi operátormi, prepíše \say{next-hop} adresu na svoj Loopback0. iBGP smerovač tak získa konektivitu do danej siete cez hraničný eBGP smerovač vo svojej oblasti. Keby sme tak neurobili, iBGP smerovač by sa nemohol dostať do siete v susednom AS, keď k nej nemá eBGP spojenie a z dôvodu bezpečnosti sme linky medzi AS neohlasovali príkazom \say{network}.


\subsubsection{Konfigurácia}
\paragraph{}
Nižšie je uvedená ukážka BGP konfigurácie na R2.

\noindent
{\fontfamily{qcr}\selectfont
\begin{small}
\begin{alltt}
R2(config)#router bgp 110
  bgp router-id 10.255.255.2
  neighbor 10.255.255.3 remote-as 110
  neighbor 10.255.255.3 update-source lo0
  neighbor 10.255.255.3 next-hop-self
  neighbor 10.255.255.4 remote-as 110
  neighbor 10.255.255.4 update-source lo0
  neighbor 10.255.255.4 next-hop-self
  neighbor 200.110.255.249 remote-as 3401
  neighbor 200.110.255.254 remote-as 4502
  network 10.255.255.2 mask 255.255.255.255
  network 200.110.0.0 mask 255.255.255.128
  aggregate-address 200.110.0.0 255.255.0.0 summary-only
  no auto-summary
  no sync
  bgp log-neighbor-changes
\end{alltt}
\end{small}
}


\subsubsection{Overenie}
\paragraph{}
Overenie konfigurácie BGP je uvedené v kapitole \ref{sumarizacia} \nameref{sumarizacia}.





\subsection{Distribúcia internetových statických smerovacích záznamov z AS3401, AS4502 a zákazníckych smerovacích záznamov z AS65001, AS5005, AS330}
\subsubsection{Popis}
\paragraph{}
Medzi autonómnymi systémami sa budú vymieňať záznamy o prefixoch z AS3401, AS4502, AS65001, AS5005 a AS330.

\subsubsection{Konfigurácia}
\paragraph{}
Nižšie je uvedená konfigurácia z R2. Druhý príkaz \say{network} ohlasuje už sumarizovanú sieť zákazníka 1 (viď kapitola \ref{sumarizacia} \nameref{sumarizacia}).

\noindent
{\fontfamily{qcr}\selectfont
\begin{small}
\begin{alltt}
R2(config)#router bgp 110
  network 10.255.255.2 mask 255.255.255.255
  network 200.110.0.0 mask 255.255.255.128
\end{alltt}
\end{small}
}

\subsubsection{Overenie}
\paragraph{}
Overenie konfigurácie BGP je uvedené v kapitole \ref{sumarizacia} \nameref{sumarizacia}.









\subsection{Sumarizácia}
\label{sumarizacia}
\subsubsection{Popis}
\paragraph{}
Verejné adresné rozsahy na Loopback100 rozhraniach sme sumarizovali pre každú AS príkazom \say{aggregate-address}. S AS 65001 vznikol problém so sumarizáciou ich verejného rozsahu na smerovačoch R3 a R4, lebo, lebo \say{Customer 1} (AS 65001) mal používal podrozsah verejných adries \say{ISP1} (AS 110). Ak by sme na R3 a R4 vykonali sumarizáciu verejných rozsahov pre Loopback100 v AS 110, spôsobilo by to, že \say{next-hop} adresa pre Loopback100 v AS 65001 by bol \say{Null0} t.j. paket by bol zahodený. Na R4 sme sumarizáciu nerobili, pretože keby vypadla linka medzi R2 a R3, do AS 65001 by sme išli cez R4, a pokiaľ by mal R4 spomenutú sumárnu cestu, vznikol by ten istý problém s konektivitou.


\subsubsection{Konfigurácia}
\paragraph{}

\noindent
{\fontfamily{qcr}\selectfont
\begin{small}
\begin{alltt}
R2(config)#router bgp 110
  aggregate-address 200.110.0.0 255.255.0.0 summary-only
\end{alltt}
\end{small}
}


\subsubsection{Overenie}
\paragraph{}
BGP, distribúciu prefixov a sumarizáciu sme overovali naraz príkazmi \say{show ip bgp}, \say{show ip route}, 

\paragraph{}
Nižšie uvádzame výpisy príkazov \say{show ip bgp} a \say{show ip route} pred odstránením liniek.

\noindent
{\fontfamily{qcr}\selectfont
\begin{small}
\begin{alltt}
R1#sh ip bgp              
BGP table version is 23, local router ID is 64.34.1.1
Status codes: s suppressed, d damped, h history, * valid, > best, i - internal,
              r RIB-failure, S Stale
Origin codes: i - IGP, e - EGP, ? - incomplete

   Network          Next Hop            Metric LocPrf Weight Path
*> 10.255.255.1/32  0.0.0.0                  0         32768 i
*  10.255.255.2/32  64.34.255.254                          0 4502 110 i
*>                  200.110.255.250          0             0 110 i
*  10.255.255.3/32  64.34.255.254                          0 4502 110 i
*>                  200.110.255.250                        0 110 i
*  10.255.255.4/32  64.34.255.254                          0 4502 110 i
*>                  200.110.255.250                        0 110 i
*  10.255.255.5/32  200.110.255.250                        0 110 4502 i
*                   200.33.255.254                         0 330 4502 i
*>                  64.34.255.254            0             0 4502 i
*  10.255.255.6/32  200.110.255.250                        0 110 4502 330 i
*                   64.34.255.254                          0 4502 330 i
*>                  200.33.255.254           0             0 330 i
*  10.255.255.7/32  200.110.255.250                        0 110 4502 330 i
*                   64.34.255.254                          0 4502 330 i
*>                  200.33.255.254                         0 330 i
*  10.255.255.8/32  64.34.255.254                          0 4502 110 65001 i
   Network          Next Hop            Metric LocPrf Weight Path
*                   200.33.255.254                         0 330 4502 110 65001 i
*>                  200.110.255.250                        0 110 65001 i
*  10.255.255.9/32  64.34.255.254                          0 4502 110 65001 i
*                   200.33.255.254                         0 330 4502 110 65001 i
*>                  200.110.255.250                        0 110 65001 i
*  10.255.255.10/32 64.34.255.254                          0 4502 110 5005 i
*                   200.110.255.250                        0 110 5005 i
*>                  200.33.255.254                         0 330 5005 i
*> 64.34.1.0/25     0.0.0.0                  0         32768 i
*  128.45.5.0/25    200.110.255.250                        0 110 4502 i
*                   200.33.255.254                         0 330 4502 i
*>                  64.34.255.254            0             0 4502 i
*  200.33.6.0/25    64.34.255.254                          0 4502 330 i
*>                  200.33.255.254           0             0 330 i
*  200.33.7.0/25    64.34.255.254                          0 4502 330 i
*>                  200.33.255.254                         0 330 i
*  200.110.0.0/25   200.33.255.254                         0 330 4502 110 i
*                   64.34.255.254                          0 4502 110 i
*>                  200.110.255.250          0             0 110 i
*  200.110.0.128/25   200.33.255.254                         0 330 4502 110 i
   Network          Next Hop            Metric LocPrf Weight Path
*                   64.34.255.254                          0 4502 110 i
*>                  200.110.255.250                        0 110 i
*> 200.110.1.0/25   200.110.255.250                        0 110 i
*                   64.34.255.254                          0 4502 110 i
*                   200.33.255.254                         0 330 4502 110 i
*  200.110.12.0/25  200.33.255.254                         0 330 4502 110 65001 i
*                   64.34.255.254                          0 4502 110 65001 i
*>                  200.110.255.250                        0 110 65001 i
*  200.110.13.0/25  200.33.255.254                         0 330 4502 110 65001 i
*                   64.34.255.254                          0 4502 110 65001 i
*>                  200.110.255.250                        0 110 65001 i
*  223.255.255.0/25 64.34.255.254                          0 4502 330 5005 i
*                   200.110.255.250                        0 110 5005 i
*>                  200.33.255.254                         0 330 5005 i


-------------------------------------------------------------------------------------


R1#sh ip route            
...

Gateway of last resort is not set

     200.110.1.0/25 is subnetted, 1 subnets
B       200.110.1.0 [20/0] via 200.110.255.250, 00:06:48
     200.33.6.0/25 is subnetted, 1 subnets
B       200.33.6.0 [20/0] via 200.33.255.254, 00:08:41
     223.255.255.0/25 is subnetted, 1 subnets
B       223.255.255.0 [20/0] via 200.33.255.254, 00:11:55
     200.33.7.0/25 is subnetted, 1 subnets
B       200.33.7.0 [20/0] via 200.33.255.254, 00:09:14
     64.0.0.0/8 is variably subnetted, 2 subnets, 2 masks
C       64.34.255.252/30 is directly connected, FastEthernet0/1
C       64.34.1.0/25 is directly connected, Loopback1
     200.110.255.0/30 is subnetted, 1 subnets
C       200.110.255.248 is directly connected, FastEthernet0/0
     200.33.255.0/30 is subnetted, 1 subnets
C       200.33.255.252 is directly connected, Serial1/0
     200.110.0.0/25 is subnetted, 1 subnets
B       200.110.0.0 [20/0] via 200.110.255.250, 00:10:45
     200.110.0.128/25 is subnetted, 1 subnets
B       200.110.0.128 [20/0] via 200.110.255.250, 00:07:52
     200.110.12.0/25 is subnetted, 1 subnets
B       200.110.12.0 [20/0] via 200.110.255.250, 00:10:14
     128.45.0.0/25 is subnetted, 1 subnets
B       128.45.5.0 [20/0] via 64.34.255.254, 00:07:22
     200.110.13.0/25 is subnetted, 1 subnets
B       200.110.13.0 [20/0] via 200.110.255.250, 00:11:15
     10.0.0.0/32 is subnetted, 10 subnets
B       10.255.255.10 [20/0] via 200.33.255.254, 00:18:19
B       10.255.255.8 [20/0] via 200.110.255.250, 00:18:19
B       10.255.255.9 [20/0] via 200.110.255.250, 00:18:19
B       10.255.255.2 [20/0] via 200.110.255.250, 00:21:55
B       10.255.255.3 [20/0] via 200.110.255.250, 00:20:22
C       10.255.255.1 is directly connected, Loopback0
B       10.255.255.6 [20/0] via 200.33.255.254, 00:19:22
B       10.255.255.7 [20/0] via 200.33.255.254, 00:18:52
B       10.255.255.4 [20/0] via 200.110.255.250, 00:19:53
B       10.255.255.5 [20/0] via 64.34.255.254, 00:19:36
\end{alltt}
\end{small}
}

\paragraph{}
Nižšie uvádzame výpisy príkazov \say{show ip bgp} a \say{show ip route} po odstránení liniek prikazmi uvedenými v časti \say{Popis}.

\noindent
{\fontfamily{qcr}\selectfont
\begin{small}
\begin{alltt}
R1#show ip bgp
BGP table version is 63, local router ID is 64.34.1.1
Status codes: s suppressed, d damped, h history, * valid, > best, i - internal,
              r RIB-failure, S Stale
Origin codes: i - IGP, e - EGP, ? - incomplete

   Network          Next Hop            Metric LocPrf Weight Path
*> 10.255.255.1/32  0.0.0.0                  0         32768 i
*  10.255.255.2/32  64.34.255.254                          0 4502 110 i
*>                  200.110.255.250          0             0 110 i
*  10.255.255.3/32  64.34.255.254                          0 4502 110 i
*>                  200.110.255.250                        0 110 i
*  10.255.255.4/32  64.34.255.254                          0 4502 110 i
*>                  200.110.255.250                        0 110 i
*  10.255.255.5/32  200.110.255.250                        0 110 4502 i
*                   200.33.255.254                         0 330 4502 i
*>                  64.34.255.254            0             0 4502 i
*  10.255.255.6/32  200.110.255.250                        0 110 4502 330 i
*                   64.34.255.254                          0 4502 330 i
*>                  200.33.255.254           0             0 330 i
*  10.255.255.7/32  200.110.255.250                        0 110 4502 330 i
*                   64.34.255.254                          0 4502 330 i
*>                  200.33.255.254                         0 330 i
*  10.255.255.8/32  64.34.255.254                          0 4502 110 65001 i
   Network          Next Hop            Metric LocPrf Weight Path
*                   200.33.255.254                         0 330 4502 110 65001 i
*>                  200.110.255.250                        0 110 65001 i
*  10.255.255.9/32  64.34.255.254                          0 4502 110 65001 i
*                   200.33.255.254                         0 330 4502 110 65001 i
*>                  200.110.255.250                        0 110 65001 i
*  10.255.255.10/32 64.34.255.254                          0 4502 110 5005 i
*                   200.110.255.250                        0 110 5005 i
*>                  200.33.255.254                         0 330 5005 i
*> 64.34.0.0/16     0.0.0.0                            32768 i
s> 64.34.1.0/25     0.0.0.0                  0         32768 i
*  128.45.0.0       200.110.255.250                        0 110 4502 i
*                   200.33.255.254                         0 330 4502 i
*>                  64.34.255.254            0             0 4502 i
*  200.33.0.0/16    64.34.255.254                          0 4502 330 i
*>                  200.33.255.254           0             0 330 i
*  200.110.0.0/16   200.33.255.254                         0 330 4502 110 i
*                   64.34.255.254                          0 4502 110 i
*>                  200.110.255.250          0             0 110 i
*  223.255.255.0    64.34.255.254                          0 4502 330 5005 i
*                   200.110.255.250                        0 110 5005 i
   Network          Next Hop            Metric LocPrf Weight Path
*>                  200.33.255.254                         0 330 5005 i


-------------------------------------------------------------------------------------


R1#show ip route
...

Gateway of last resort is not set

B    223.255.255.0/24 [20/0] via 200.33.255.254, 00:32:37
     64.0.0.0/8 is variably subnetted, 3 subnets, 3 masks
C       64.34.255.252/30 is directly connected, FastEthernet0/1
B       64.34.0.0/16 [200/0] via 0.0.0.0, 00:28:12, Null0
C       64.34.1.0/25 is directly connected, Loopback1
     200.110.255.0/30 is subnetted, 1 subnets
C       200.110.255.248 is directly connected, FastEthernet0/0
     200.33.255.0/30 is subnetted, 1 subnets
C       200.33.255.252 is directly connected, Serial1/0
B    128.45.0.0/16 [20/0] via 64.34.255.254, 00:27:42
     10.0.0.0/32 is subnetted, 10 subnets
B       10.255.255.10 [20/0] via 200.33.255.254, 00:54:15
B       10.255.255.8 [20/0] via 200.110.255.250, 00:54:15
B       10.255.255.9 [20/0] via 200.110.255.250, 00:54:16
B       10.255.255.2 [20/0] via 200.110.255.250, 00:57:51
B       10.255.255.3 [20/0] via 200.110.255.250, 00:56:19
C       10.255.255.1 is directly connected, Loopback0
B       10.255.255.6 [20/0] via 200.33.255.254, 00:55:18
B       10.255.255.7 [20/0] via 200.33.255.254, 00:54:47
B       10.255.255.4 [20/0] via 200.110.255.250, 00:55:48
B       10.255.255.5 [20/0] via 64.34.255.254, 00:55:32
B    200.33.0.0/16 [20/0] via 200.33.255.254, 00:28:44
B    200.110.0.0/16 [20/0] via 200.110.255.250, 00:27:13
\end{alltt}
\end{small}
}

\paragraph{}
Z druhého výpisu BGP tabuľky vidíme, že počet prefixov sa v dôsledku sumarizácie sietí znížil. Z druhého výpisu smerovacej tabuľky vidíme, že interné adresy autonómnych systémov nie sú ohlasované.





\subsection{Prepísať privátne AS65001}
\subsubsection{Popis}
\paragraph{}
Na konci atribútu AS\_PATH máme pre 10.255.255.8 privátne číslo AS. ISP1 musí privátne číslo AS odstrániť, lebo také sa nemôžu dostať do internetu, inak by sa narušilo smerovanie. Odstraňovanie sa bude diať na smerovačoch R2 (v smere k R1 a R5) a R4 (v smere k R10), pretože R2 a R4 tie sú hraničné smerovače v AS 110.



\subsubsection{Konfigurácia}
\paragraph{}
Na smerovačoch R2 a R4 vykonáme tieto príkazy (v globálnom konfiguračnom móde):

\noindent
{\fontfamily{qcr}\selectfont
\begin{small}
\begin{alltt}
R2

router bgp 110
neighbor 200.110.255.249 remove-private-as
neighbor 200.110.255.254 remove-private-as

------------------------------------------

R4

router bgp 110
neighbor 200.110.255.246 remove-private-as
\end{alltt}
\end{small}
}


\subsubsection{Overenie}
\paragraph{}
Prikazom \say{show ip bgp overit} sme overili, že atribút AS\_PATH neobsahuje priváte číslo zákazníka 1.



\noindent
{\fontfamily{qcr}\selectfont
\begin{small}
\begin{alltt}
R1

Predtým...

R1#show ip bgp 10.255.255.8
BGP routing table entry for 10.255.255.8/32, version 154
Paths: (2 available, best #1, table Default-IP-Routing-Table)
  Advertised to update-groups:
        1
  110 65001
    200.110.255.250 from 200.110.255.250 (10.255.255.2)
      Origin IGP, localpref 100, valid, external, best
  4502 110 65001
    64.34.255.254 from 64.34.255.254 (10.255.255.5)
      Origin IGP, localpref 100, valid, external


----------------------------------------------------------------


Potom...

R1#show ip bgp 10.255.255.8
BGP routing table entry for 10.255.255.8/32, version 157
Paths: (2 available, best #1, table Default-IP-Routing-Table)
Flag: 0x820
  Advertised to update-groups:
        1
  110
    200.110.255.250 from 200.110.255.250 (10.255.255.2)
      Origin IGP, localpref 100, valid, external, best
  4502 110
    64.34.255.254 from 64.34.255.254 (10.255.255.5)
      Origin IGP, localpref 100, valid, external



================================================================


R10

Predtým...

R10#show ip bgp 10.255.255.8
BGP routing table entry for 10.255.255.8/32, version 56
Paths: (2 available, best #2, table Default-IP-Routing-Table)
  Not advertised to any peer
  330 3401 110
    200.33.255.245 from 200.33.255.245 (10.255.255.7)
      Origin IGP, localpref 100, valid, external
  110 65001
    200.110.255.245 from 200.110.255.245 (10.255.255.4)
      Origin IGP, localpref 100, valid, external, best


----------------------------------------------------------------

Potom ...

R10#show ip bgp 10.255.255.8
BGP routing table entry for 10.255.255.8/32, version 59
Paths: (2 available, best #2, table Default-IP-Routing-Table)
Flag: 0x820
  Not advertised to any peer
  330 3401 110
    200.33.255.245 from 200.33.255.245 (10.255.255.7)
      Origin IGP, localpref 100, valid, external
  110
    200.110.255.245 from 200.110.255.245 (10.255.255.4)
      Origin IGP, localpref 100, valid, external, best
\end{alltt}
\end{small}
}


\paragraph{}
Z výpisov AS\_PATH vidíme, že privátny AS 65001 nie je ohlasovaný do internetu ani cez R2, ani cez R4.


\subsection{Kontrola konektivity medzi zákazníckymi a internetovými smerovacími záznamami}
\subsubsection{Overenie}
\paragraph{}

\noindent
{\fontfamily{qcr}\selectfont
\begin{small}
\begin{alltt}
Type escape sequence to abort.
Sending 5, 100-byte ICMP Echos to 10.255.255.1, timeout is 2 seconds:
Packet sent with a source address of 10.255.255.5 
!!!!!
Success rate is 100 percent (5/5), round-trip min/avg/max = 16/17/20 ms
Type escape sequence to abort.
Sending 5, 100-byte ICMP Echos to 10.255.255.5, timeout is 2 seconds:
!!!!!
Success rate is 100 percent (5/5), round-trip min/avg/max = 1/1/1 ms
Type escape sequence to abort.
Sending 5, 100-byte ICMP Echos to 10.255.255.2, timeout is 2 seconds:
Packet sent with a source address of 10.255.255.5 
!!!!!
Success rate is 100 percent (5/5), round-trip min/avg/max = 8/25/44 ms
Type escape sequence to abort.
Sending 5, 100-byte ICMP Echos to 10.255.255.5, timeout is 2 seconds:
!!!!!
Success rate is 100 percent (5/5), round-trip min/avg/max = 1/1/1 ms
Type escape sequence to abort.
Sending 5, 100-byte ICMP Echos to 10.255.255.3, timeout is 2 seconds:
Packet sent with a source address of 10.255.255.5 
!!!!!
Success rate is 100 percent (5/5), round-trip min/avg/max = 36/49/68 ms
Type escape sequence to abort.
Sending 5, 100-byte ICMP Echos to 10.255.255.5, timeout is 2 seconds:
!!!!!
Success rate is 100 percent (5/5), round-trip min/avg/max = 1/1/4 ms
Type escape sequence to abort.
Sending 5, 100-byte ICMP Echos to 10.255.255.4, timeout is 2 seconds:
Packet sent with a source address of 10.255.255.5 
!!!!!
Success rate is 100 percent (5/5), round-trip min/avg/max = 28/48/72 ms
Type escape sequence to abort.
Sending 5, 100-byte ICMP Echos to 10.255.255.5, timeout is 2 seconds:
!!!!!
Success rate is 100 percent (5/5), round-trip min/avg/max = 1/1/4 ms
Type escape sequence to abort.
Sending 5, 100-byte ICMP Echos to 10.255.255.6, timeout is 2 seconds:
Packet sent with a source address of 10.255.255.5 
!!!!!
Success rate is 100 percent (5/5), round-trip min/avg/max = 16/28/32 ms
Type escape sequence to abort.
Sending 5, 100-byte ICMP Echos to 10.255.255.5, timeout is 2 seconds:
!!!!!
Success rate is 100 percent (5/5), round-trip min/avg/max = 1/1/4 ms
Type escape sequence to abort.
Sending 5, 100-byte ICMP Echos to 10.255.255.7, timeout is 2 seconds:
Packet sent with a source address of 10.255.255.5 
!!!!!
Success rate is 100 percent (5/5), round-trip min/avg/max = 40/48/60 ms
Type escape sequence to abort.
Sending 5, 100-byte ICMP Echos to 10.255.255.5, timeout is 2 seconds:
!!!!!
Success rate is 100 percent (5/5), round-trip min/avg/max = 1/1/1 ms
Type escape sequence to abort.
Sending 5, 100-byte ICMP Echos to 10.255.255.8, timeout is 2 seconds:
Packet sent with a source address of 10.255.255.5 
!!!!!
Success rate is 100 percent (5/5), round-trip min/avg/max = 56/66/84 ms
Type escape sequence to abort.
Sending 5, 100-byte ICMP Echos to 10.255.255.5, timeout is 2 seconds:
!!!!!
Success rate is 100 percent (5/5), round-trip min/avg/max = 1/1/1 ms
Type escape sequence to abort.
Sending 5, 100-byte ICMP Echos to 10.255.255.9, timeout is 2 seconds:
Packet sent with a source address of 10.255.255.5 
!!!!!
Success rate is 100 percent (5/5), round-trip min/avg/max = 80/92/104 ms
Type escape sequence to abort.
Sending 5, 100-byte ICMP Echos to 10.255.255.5, timeout is 2 seconds:
!!!!!
Success rate is 100 percent (5/5), round-trip min/avg/max = 1/1/4 ms
Type escape sequence to abort.
Sending 5, 100-byte ICMP Echos to 10.255.255.10, timeout is 2 seconds:
Packet sent with a source address of 10.255.255.5 
!!!!!
Success rate is 100 percent (5/5), round-trip min/avg/max = 48/65/84 ms
Type escape sequence to abort.
Sending 5, 100-byte ICMP Echos to 10.255.255.5, timeout is 2 seconds:
!!!!!
Success rate is 100 percent (5/5), round-trip min/avg/max = 1/1/1 ms
\end{alltt}
\end{small}
}


\newpage

\subsection{ISP politika}
\paragraph{}
ISP politika umožňuje ovplyvňovať smerovanie prevádzky medzi BGP autonómnymi systémami. Používame na to BGP atribúty , ktoré vieme meniť. Použité atribúty vysvetľujeme bližšie pri jednotlivých úlohách.

\paragraph{}
Linku ku upstream providerovi sme sa vo všetkých úlohách snažili používať minimálne. Namiesto toho sme využívali pre ISP 1 a 2 peeringové centrum. Toto správanie sme zapezpečili pomocou zmeny \say{local preferenice} na R2 a R6 nasledovne:

\noindent
{\fontfamily{qcr}\selectfont
\begin{small}
\begin{alltt}
!R6
route-map R1-R6 permit 10
  set local-preference 60
route-map R5-R6 permit 10
  set local-preference 70
router bgp 330
  neighbor 200.33.255.249 route-map R5-R6 in
  neighbor 200.33.255.253 route-map R1-R6 in


-------------------------------------------------------------


!R2
route-map R1-R2 permit 10
  set local-preference 60
route-map R5-R2 permit 10
  set local-preference 70
router bgp 110
  neighbor 200.110.255.249 route-map R1-R2 in
  neighbor 200.110.255.254 route-map R5-R2 in
\end{alltt}
\end{small}
}

\subsection{Primárna linka R3−R8}
\subsubsection{Popis}
\paragraph{}
Úlohou bolo zabezpečiť aby bola pre smerovač R9 zvolená v oboch smeroch primárna linka do AS 110 cez smerovač R3 a nie R4 (viď obr. \ref{fig:bgp_isis_topo_r3_r8}). 

\paragraph{}
Na R8 a R9 sme vytvorili routemapu s názvom \say{AS\_KAM\_TO\_IDE:AS\_ODKIAL\_TO\_IDE}. Na zmenu smerovania smerom von z AS 65001 sme použili atribút \say{Local Preference}, ktorý sme nastavili routemapou na smerovačoch R8 a R9. Chceme aby aj R9 išla cez R8, nie cez R4. Preto si R8 si musí zdvihnúť \say{local preference} smerom na R9, a ohlasovať zvýšenú \say{local preference} na R9. Pri akejkoľvek zmene politiky musíme reštartovať BGP proces, aby sa prejavili zmeny.

\paragraph{}
Keď si teraz zobrazíme informácie o prefixe 10.255.255.8 príkazom \say{show ip bgp 10.255.255.8}, vidíme, že číslo komunity "community" je nejake vysoke cislo. Aby sme mohli precitat cislo community, na vsetkych 4 routroch v globalnom konfig. rezime musíme na smerovačoch R3 a R4 zmeniť formát, v akom sa číslo komunity zobrazuje príkazom \say{ip bgp-community new-format} v globálnom konfiguračnom režime.

\paragraph{}
LENZE to je iba riesenie v smere z AS 65001 VON. Teraz musime zabezpecit spatny smer DO AS 65001
da sa to riesit MEDom alebo cez routemapy a nastavovanie local preference.
r8 a r9 routemapa out na susedne routre v AS 110
-vsetko sa ma riesit komunitami
-zvysime routemapou local preference na R3 v smere IN (lebo sa musime pozerat odkial idu updaty) na suseda R8. dat tam match na community, ked sa zhoduje s 65001:110
-na R3 bude routemapa CUSTOMER

\paragraph{}
ACLko \say{PREFER} sme mohli na R8 dať buď na suseda R3 v smere IN, alebo na suseda R9 v smere OUT, ale v takom prípade, keby sa do AS 65001 pridal další smerovač pripojený k R8, museli by sme routemapu aplikovať aj na rozhranie na R8, ktorým by bol nový smerovač pripojený ku R8.

\begin{figure}[!htbp]
\centering
\includegraphics[width=14cm,keepaspectratio]{bgp_isis_r3_r8_primary}
\caption{Preferovaná linka R3-R8}
\label{fig:bgp_isis_topo_r3_r8}
\end{figure}

\newpage

\subsubsection{Konfigurácia}
\paragraph{}

\noindent
{\fontfamily{qcr}\selectfont
\begin{small}
\begin{alltt}
=============================================================
!SMEROM VON
=============================================================

!R8
conf t
ip bgp-community new-format
route-map OUT permit 10
set community 65001:110
router bgp 65001
neighbor 200.110.255.241 route-map OUT out
neighbor 200.110.255.241 send-community

!Nastavenie priority pre R3 smerom von
route-map PREFER permit 10
  set local-preference 110
route-map OUT permit 10
  set community 65001:110
router bgp 65001
  neighbor 200.110.255.241 route-map PREFER in
  neighbor 200.110.255.241 route-map OUT out
end
clear ip bgp * in
clear ip bgp * out


-------------------------------------------------------------


!R3, R4
ip bgp-community new-format



=============================================================
!SMEROM DNU
=============================================================

!R3
!vytvoríme ACL pre komunitu
ip community-list 1 permit 65001:110

route-map CUSTOMER permit 10
  match community 1
  set local-preference 110
router bgp 110
  neighbor 200.110.255.242 route-map CUSTOMER in
end
clear ip bgp * in
\end{alltt}
\end{small}
}

\subsubsection{Overenie}
\paragraph{}
Zmenu smerovania sme overovali príkazmi \say{show ip bgp} a \say{traceroute}.

\noindent
{\fontfamily{qcr}\selectfont
\begin{small}
\begin{alltt}
R9# show ip bgp
BGP table version is 36, local router ID is 200.110.13.1
Status codes: s suppressed, d damped, h history, * valid, > best, i - internal,
              r RIB-failure, S Stale
Origin codes: i - IGP, e - EGP, ? - incomplete

   Network          Next Hop            Metric LocPrf Weight Path
*>i0.0.0.0          10.255.255.8             0    110      0 110 i
*                   200.110.255.237          0             0 110 i
*>i10.255.255.2/32  10.255.255.8             0    110      0 110 i
*                   200.110.255.237                        0 110 i
*  10.255.255.3/32  200.110.255.237                        0 110 i
*>i                 10.255.255.8             0    110      0 110 i
*>i10.255.255.4/32  10.255.255.8             0    110      0 110 i
*                   200.110.255.237          0             0 110 i
r>i10.255.255.8/32  10.255.255.8             0    100      0 i
*> 10.255.255.9/32  0.0.0.0                  0         32768 i
*>i10.255.255.10/32 10.255.255.8             0    110      0 110 5005 i
*                   200.110.255.237                        0 110 5005 i
*>i200.110.0.0/16   10.255.255.8             0    110      0 110 i
*                   200.110.255.237                        0 110 i
*  200.110.3.0/25   200.110.255.237                        0 110 i
*>i                 10.255.255.8             0    110      0 110 i
r>i200.110.12.0/25  10.255.255.8             0    100      0 i
   Network          Next Hop            Metric LocPrf Weight Path
*>i223.255.255.0    10.255.255.8             0    110      0 110 5005 i
*                   200.110.255.237                        0 110 5005 i


-------------------------------------------------------------


R9#traceroute 10.255.255.4 source 10.255.255.9

Type escape sequence to abort.
Tracing the route to 10.255.255.4

  1 10.110.89.8 [AS 110] 16 msec 20 msec 16 msec
  2 200.110.255.241 [AS 110] 36 msec 36 msec 40 msec
  3 10.110.34.4 [AS 110] 36 msec *  24 msec



=============================================================



R4#sh ip bgp
BGP table version is 36, local router ID is 200.110.4.1
Status codes: s suppressed, d damped, h history, * valid, > best, i - internal,
              r RIB-failure, S Stale
Origin codes: i - IGP, e - EGP, ? - incomplete

   Network          Next Hop            Metric LocPrf Weight Path
*>i10.255.255.1/32  10.255.255.2             0    110      0 3401 i
r>i10.255.255.2/32  10.255.255.2             0    100      0 i
r>i10.255.255.3/32  10.255.255.3             0    100      0 i
*> 10.255.255.4/32  0.0.0.0                  0         32768 i
*>i10.255.255.5/32  10.255.255.2             0    110      0 3401 4502 i
*>i10.255.255.6/32  10.255.255.2             0    110      0 3401 330 i
*>i10.255.255.7/32  10.255.255.2             0    110      0 3401 330 i
*>i10.255.255.8/32  10.255.255.3             0    110      0 65001 i
*                   200.110.255.238                        0 65001 i
*>i10.255.255.9/32  10.255.255.3             0    110      0 65001 i
*                   200.110.255.238          0             0 65001 i
*>i64.34.0.0/16     10.255.255.2             0    110      0 3401 i
*>i128.45.0.0       10.255.255.2             0    110      0 3401 4502 i
*>i200.33.0.0/16    10.255.255.2             0    110      0 3401 330 i
*>i200.110.0.0/16   10.255.255.2             0    100      0 i
*> 200.110.4.0/25   0.0.0.0                  0         32768 i
*>i200.110.12.0/25  10.255.255.3             0    110      0 65001 i
   Network          Next Hop            Metric LocPrf Weight Path
*                   200.110.255.238                        0 65001 i


-------------------------------------------------------------


R4#traceroute 10.255.255.9 source 10.255.255.4

Type escape sequence to abort.
Tracing the route to 10.255.255.9

  1 10.110.34.3 12 msec 16 msec 16 msec
  2 200.110.255.242 36 msec 36 msec 36 msec
  3 10.110.89.9 36 msec *  64 msec

\end{alltt}
\end{small}
}







\subsection{Primárna linka R4−R10}
\subsubsection{Popis}
\paragraph{}
Záklazník 2 chce primárne využívať linku ku R4 pre vstup aj výstup t.j. ping z R5 na R7 nepôjde cez AS 330 (R5 -\textgreater{} R6 -\textgreater{} R7), ale cez \textbf{R5 -\textgreater{} R2 -\textgreater{} R4 -\textgreater{} R10 -\textgreater{} R7}. Rovnako aj v spätnom smere.

\begin{figure}[!htbp]
\centering
\includegraphics[width=14cm,keepaspectratio]{bgp_isis_r4_r10_primary}
\caption{Preferovaná linka R4-R10}
\label{fig:bgp_isis_topo}
\end{figure}

\subsubsection{Konfigurácia}
\paragraph{}
Konfigurujeme smerovač R7.

\noindent
{\fontfamily{qcr}\selectfont
\begin{small}
\begin{alltt}
=============================================================
!SMEROM VON
=============================================================

!R7
route-map ROUTE1 permit 10
  match as-path 1
  match community 2
  set local-preference 65
exit
router bgp 330
neighbor 200.33.255.246 route-map ROUTE1 in


=============================================================
!SMEROM DNU
=============================================================


!R10
route-map R4PREF permit 10
  set local-preference 160
router bgp 5005
  neighbor 200.110.255.245 route-map R4PREF in
\end{alltt}
\end{small}
}


\subsubsection{Overenie}
\paragraph{}
Primárnu linku sme overovali príkazom \say{show ip bgp} na R7 a R10.

\noindent
{\fontfamily{qcr}\selectfont
\begin{small}
\begin{alltt}
R10#sh ip bgp
BGP table version is 127, local router ID is 10.255.255.10
Status codes: s suppressed, d damped, h history, * valid, > best, i - internal,
              r RIB-failure, S Stale
Origin codes: i - IGP, e - EGP, ? - incomplete

   Network          Next Hop            Metric LocPrf Weight Path
*> 10.255.255.1/32  200.110.255.245               160      0 110 3401 i
*                   200.33.255.245                         0 330 3401 i
*> 10.255.255.2/32  200.110.255.245               160      0 110 i
*                   200.33.255.245                         0 330 3401 110 i
*  10.255.255.3/32  200.33.255.245                         0 330 3401 110 i
*>                  200.110.255.245               160      0 110 i
*  10.255.255.4/32  200.33.255.245                         0 330 3401 110 i
*>                  200.110.255.245          0    160      0 110 i
*> 10.255.255.5/32  200.110.255.245               160      0 110 3401 4502 i
*                   200.33.255.245                         0 330 3401 4502 i
*> 10.255.255.6/32  200.110.255.245               160      0 110 3401 330 i
*                   200.33.255.245                         0 330 i
*> 10.255.255.7/32  200.110.255.245               160      0 110 3401 330 i
*                   200.33.255.245           0             0 330 i
*> 10.255.255.8/32  200.110.255.245               160      0 110 i
*                   200.33.255.245                         0 330 3401 110 i
*  10.255.255.9/32  200.33.255.245                         0 330 3401 110 i


=============================================================


R7#sh ip bgp
BGP table version is 23, local router ID is 10.255.255.7
Status codes: s suppressed, d damped, h history, * valid, > best, 
              i - internal, r RIB-failure, S Stale
Origin codes: i - IGP, e - EGP, ? - incomplete

   Network          Next Hop        Metric LocPrf Weight Path
*>i10.255.255.1/32  10.255.255.6         0    100      0 3401 i
*>i10.255.255.2/32  10.255.255.6         0    100      0 4502 110 i
*>i10.255.255.3/32  10.255.255.6         0    100      0 3401 110 i
*>i10.255.255.4/32  10.255.255.6         0    100      0 4502 110 i
*>i10.255.255.5/32  10.255.255.6         0    100      0 4502 i
r>i10.255.255.6/32  10.255.255.6         0    100      0 i
*> 10.255.255.7/32  0.0.0.0              0         32768 i
*>i10.255.255.8/32  10.255.255.6         0    100      0 4502 110 65001 i
*>i10.255.255.9/32  10.255.255.6         0    100      0 4502 110 65001 i
*>i10.255.255.10/32 10.255.255.6         0    100      0 3401 110 5005 i
*>i64.34.0.0/16     10.255.255.6         0    100      0 3401 i
*>i128.45.0.0       10.255.255.6         0    100      0 4502 i
* i200.33.0.0/16    10.255.255.6         0    100      0 i
*>                  0.0.0.0                        32768 i
s> 200.33.7.0/25    0.0.0.0              0         32768 i
*>i200.110.0.0/16   10.255.255.6         0    100      0 4502 110 i
*>i223.255.255.0    10.255.255.6         0    100      0 4502 110 5005 i
\end{alltt}
\end{small}
}

\paragraph{}
Môžeme si všimnúť, že prevádzka ide cez R4, kde sme nastavili \say{local preference} na 160. Smerovač R7 presmeroval prevádzku cez R6.




\subsection{Distribuovať iba default, AS5005 a peering prefixy do AS65001}
\subsubsection{Popis}
\paragraph{}
Potrebujeme na R1 nastaviť komunitu. Namiesto toho, aby sme videli videl siete, ktoré sú z R1, uvidíme default route.

\subsubsection{Konfigurácia}
\paragraph{}
Konfigurovali sme smerovače R1, R3 a R4. R1 posiela komunitu, R3 a R4 túto komunitu zahadzuje smerom na zákazníka 1.

\noindent
{\fontfamily{qcr}\selectfont
\begin{small}
\begin{alltt}
!R1
route-map COM permit 10
  set community 3401:65001
router bgp 3401
neighbor 200.110.255.250 send-community
neighbor 200.33.255.254 send-community
  neighbor 200.110.255.250 route-map COM out
  neighbor 200.33.255.254 route-map COM out


=============================================================


!R2
R2(config)#router bgp 110
  neighbor 10.255.255.3 send-community
  neighbor 10.255.255.4 send-community
do clear ip bgp * out


=============================================================


!R3
R3(config)#ip community-list 1 permit 3401:65001
route-map DEFAULT deny 10
  match community 1
  exit
route-map DEFAULT permit 20
router bgp 110
  neighbor 200.110.255.242 route-map DEFAULT out
  neighbor 200.110.255.242 default-originate


=============================================================


!R4
ip community-list 1 permit 3401:65001
route-map DEFAULT deny 10
  match community 1
  exit
route-map DEFAULT permit 20
exit
router bgp 110
  neighbor 200.110.255.238 route-map DEFAULT out
  neighbor 200.110.255.238 default-originate

R4(config-router)#do clear ip bgp * out


=============================================================


!R9
ip bgp-community new-format
route-map OUT permit 10
set community 65001:110
router bgp 65001
neighbor 10.255.255.8 route-map OUT out
neighbor 10.255.255.8 send-community
!neighbor 200.110.255.237 route-map OUT out
!neighbor 200.110.255.237 send-community
end
clear ip bgp * out

\end{alltt}
\end{small}
}


\subsubsection{Overenie}
\paragraph{}
Default route ku zákazníkovi 1 sme overovali výpisom smerovacej tabuľky a BGP databázy.

\noindent
{\fontfamily{qcr}\selectfont
\begin{small}
\begin{alltt}
R8#sh ip route
...

Gateway of last resort is 200.110.255.241 to network 0.0.0.0

     200.110.4.0/25 is subnetted, 1 subnets
B       200.110.4.0 [20/0] via 200.110.255.241, 01:20:11
B    223.255.255.0/24 [20/0] via 200.110.255.241, 00:34:49
     200.110.255.0/30 is subnetted, 2 subnets
C       200.110.255.240 is directly connected, FastEthernet0/0
i L2    200.110.255.236 [115/10] via 10.110.89.9, FastEthernet0/1
     200.110.12.0/25 is subnetted, 1 subnets
C       200.110.12.0 is directly connected, Loopback1
     200.110.13.0/25 is subnetted, 1 subnets
i L2    200.110.13.0 [115/10] via 10.110.89.9, FastEthernet0/1
     10.0.0.0/8 is variably subnetted, 7 subnets, 2 masks
B       10.255.255.10/32 [20/0] via 200.110.255.241, 00:34:50
C       10.255.255.8/32 is directly connected, Loopback0
i L2    10.255.255.9/32 [115/10] via 10.110.89.9, FastEthernet0/1
B       10.255.255.2/32 [20/0] via 200.110.255.241, 01:20:14
B       10.255.255.3/32 [20/0] via 200.110.255.241, 01:20:14
B       10.255.255.4/32 [20/0] via 200.110.255.241, 01:20:14
C       10.110.89.0/24 is directly connected, FastEthernet0/1
B*   0.0.0.0/0 [20/0] via 200.110.255.241, 00:08:57
B    200.110.0.0/16 [20/0] via 200.110.255.241, 01:20:14


-------------------------------------------------------------


R8#sh ip bgp                            
BGP table version is 132, local router ID is 200.110.12.1
Status codes: s suppressed, d damped, h history, * valid, > best, i - internal,
              r RIB-failure, S Stale
Origin codes: i - IGP, e - EGP, ? - incomplete

   Network          Next Hop            Metric LocPrf Weight Path
*> 0.0.0.0          200.110.255.241          0    110      0 110 i
*> 10.255.255.2/32  200.110.255.241               110      0 110 i
*> 10.255.255.3/32  200.110.255.241          0    110      0 110 i
*> 10.255.255.4/32  200.110.255.241               110      0 110 i
*> 10.255.255.8/32  0.0.0.0                  0         32768 i
r>i10.255.255.9/32  10.255.255.9             0    100      0 i
*> 10.255.255.10/32 200.110.255.241               110      0 110 5005 i
*> 200.110.0.0/16   200.110.255.241               110      0 110 i
*> 200.110.4.0/25   200.110.255.241               110      0 110 i
*> 200.110.12.0/25  0.0.0.0                  0         32768 i
*> 223.255.255.0    200.110.255.241               110      0 110 5005 i
\end{alltt}
\end{small}
}

\paragraph{}
Z výpisu smerovacej tabuľky a BGP databázy z R8 vidíme, že na R8 nevidíme siete, ktoré sú z R1. Namiesto toho vidíme default route z R3 resp. R4.


\subsection{AS5005 nesme byť nikdy transit}
\subsubsection{Popis}
\paragraph{}
Je nežiadúce, aby komunikácia medzi poskytovateľmi \say{ISP 1} a \say{ISP 2} prechádzala cez oblasť 5005 (Customer 2). Jednak by to bolo bezpečnostné riziko, a navyše by sme obmedzovali prenosovú rýchlosť zákazníka 2.

\subsubsection{Konfigurácia}
\paragraph{}

\noindent
{\fontfamily{qcr}\selectfont
\begin{small}
\begin{alltt}
!R10
ip as-path access-list 1 permit ^$
route-map OUT_110 permit 10
  match as-path 1
  set community 5005:110
route-map OUT_330 permit 20
  match as-path 1
  set community 5005:330
router bgp 5005
  neighbor 200.110.255.245 route-map OUT_110 out
  neighbor 200.33.255.245 route-map OUT_330 out

\end{alltt}
\end{small}
}



\subsection{Peering iba pre ISP1 a ISP2, nie pre prefixy naučené z Upstream ISP}
\subsubsection{Popis}
\paragraph{}
V tejto úlohe je potrebné zmeniť smerovanie tak, aby sa na Upstream providera nebolo možné dostať cez peeringové centrum, ale iba ISP 1 a 2.

\subsubsection{Konfigurácia}
\paragraph{}
Konfigurovali sme smerovače R2 a R6.

\noindent
{\fontfamily{qcr}\selectfont
\begin{small}
\begin{alltt}
!R2
route-map DENYR1 deny 10
  match community 1
route-map DENYR1 permit 20
router bgp 110
  neighbor 200.110.255.254 route-map DENYR1 out
clear


!R6
ip community-list 1 permit 3401:65001
route-map DENYR1 deny 10
 match community 1
route-map DENYR1 permit 20
router bgp 330
 nei 200.33.255.249 route-map DENYR1 out
do clear ip bgp * out
\end{alltt}
\end{small}
}

\paragraph{}
Teraz bol problém, že ping šiel od R10 do R4 R2 R5 R1, preto bolo treba zadať do R2 takúto routemapu:

\noindent
{\fontfamily{qcr}\selectfont
\begin{small}
\begin{alltt}
!R2
route-map R1-R2 permit 5
 match community 1
 set local-preference 110
\end{alltt}
\end{small}
}

\subsubsection{Overenie}
\paragraph{}
Konfiguráciu sme overovali príkazom \say{traceroute} z R10 do R5.

\noindent
{\fontfamily{qcr}\selectfont
\begin{small}
\begin{alltt}
R10#traceroute 10.255.255.5 source 10.255.255.10

Type escape sequence to abort.
Tracing the route to 10.255.255.5

  1 200.110.255.245 [AS 110] 4 msec 16 msec 20 msec
  2 10.110.24.2 40 msec 40 msec 44 msec
  3 200.110.255.249 [AS 110] 36 msec 76 msec 52 msec
  4 64.34.255.254 [AS 3401] 60 msec *  68 msec

\end{alltt}
\end{small}
}

\paragraph{}
Vidíme, že smerovanie išlo po trase \textbf{R10 -\textgreater{} R4 -\textgreater{} R2 -\textgreater{} R1}.

\subsection{Overiť funkčnosť nastavenia politiky vhodnými výpadkami liniek a smerovačov}
\subsubsection{Popis}
\paragraph{}
Rozhodli sme sa, že odpojíme linky R3-R8, R1-R2

\subsubsection{Overenie}
\paragraph{}
Výpadok linky R3-R8.

\noindent
{\fontfamily{qcr}\selectfont
\begin{small}
\begin{alltt}
R8#traceroute 10.255.255.3 source 10.255.255.8

Type escape sequence to abort.
Tracing the route to 10.255.255.3

  1 10.110.89.9 [AS 110] 16 msec 24 msec 16 msec
  2 200.110.255.237 [AS 110] 36 msec 36 msec 40 msec
  3 10.110.34.3 [AS 110] 60 msec *  76 msec

...

*Mar 20 11:39:59.651: %BGP-5-ADJCHANGE: neighbor 200.110.255.241 Up  * 

...

R8#traceroute 10.255.255.3 source 10.255.255.8

Type escape sequence to abort.
Tracing the route to 10.255.255.3

  1 200.110.255.241 [AS 110] 28 msec *  32 msec
\end{alltt}
\end{small}
}

\paragraph{}
Vidíme, že premávka bola presmerovaná z \textbf{R8 -\textgreater{} R3} na \textbf{R8 -\textgreater{} R9 -\textgreater{} R4 -\textgreater{} R3}.

\paragraph{}
Výpadok linky R1-R2.

\noindent
{\fontfamily{qcr}\selectfont
\begin{small}
\begin{alltt}
R10#traceroute 10.255.255.5 source 10.255.255.10

Type escape sequence to abort.
Tracing the route to 10.255.255.5

  1 200.110.255.245 [AS 110] 16 msec 16 msec 16 msec
  2 10.110.24.2 40 msec 36 msec 36 msec
  3 200.110.255.249 [AS 110] 68 msec 64 msec 60 msec
  4 64.34.255.254 [AS 3401] 68 msec *  72 msec
R10#traceroute 10.255.255.5 source 10.255.255.10

Type escape sequence to abort.
Tracing the route to 10.255.255.5

  1 200.110.255.245 [AS 110] 8 msec 12 msec 20 msec
  2 10.110.24.2 36 msec 36 msec 40 msec
  3 200.110.255.254 [AS 110] 36 msec *  64 msec
\end{alltt}
\end{small}
}

\paragraph{}
Vidíme, že premávka bola presmerovaná z \textbf{R10 -\textgreater{} R4 -\textgreater{} R2 -\textgreater{} R1 -\textgreater{} R5} na \textbf{R10 -\textgreater{} R4 -\textgreater{} R2 -\textgreater{} R5}.

\subsection{Otázky}
\paragraph{}
1. Čo je to BGP Routing Information Base (RIB)?\\
A. Tabuľka topológie BGP, ktorá obsahuje informácie o NLRI naucené z BGP +\\
B. BGP tabulka susedov\\
C. BGP politiky smerovania pre všetky NLRI naučené z BGP\\
D. BGP tabulka oblastí \\

2. Ako sa používa cieľová adresa IP na urcenie a dosiahnutie susedov?\\
A. Pomocou atribútu NEXT\_HOP propagovaného susedom\\
B. Manuálne nakonfigurovaný pomocou príkazu neighbor +\\
C. Odošle sa HELLO správa \\
D. Definujte v skupine partnerov konfederácie\\

3. V predvolenom nastavení BGP používa na výber najlepšej trasy?\\
A. najnižšiu metriku\\
B. najnižšiu Local Preference\\
C. najkratší AS Path +\\
D. najvacší Milti-Exit Discrimintor\\

4. Ako BGP komunikuje so svojimi susedmi ?\\
A. používa multicast adresu 224.0.0.2 na UDP port 1985\\
B. používa multicast adresu 224.0.0.2 na UDP port 3222\\
C. používa unicast adresu suseda na UDP port 520\\
D. používa unicast adresu suseda na TCP port 179 +\\

5. Aký typ správy BGP slúži na výmenu NLRI ?\\
A. Hello\\
B. Update +\\
C. Notification\\
D. Keepalive\\

\end{document}